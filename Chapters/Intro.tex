\chapter{Introduction}
\section{Preface}
Multi-Agent PathFinding(MAPF) is a problem which computes feasible plans for many agents cooperatively.
MAPF has been popular in the research community for many decades due to its wide variaty of applications, including automated warehouse\cite{warehouse1_2008,warehouse2_2020,warehouse3_2017}, Robot Coordination\cite{robot_2016}, multi-drone coordination\cite{drone_2022}, Video Games\cite{vidGame_2022} and robot arm motion planning \cite{motion_planning_2021}.
However, the way classical MAPF problem was defined is too abstract to be able to catch any practical nuances.

\section{Research Question and Objective}
The over-arching research objective for this PhD program is to extend the classical abstract MAPF problem model to capture practical considerations.
The first one is to considerating continuous time execution.

\subsection{Research Questions}
\paragraph*{RQ1:} Can we solve Continuous Time MAPF, or Continuous Space MAPF optimally?
\paragraph*{RQ2:} How to react quickly when interruption appears during the live execution?
\paragraph*{RQ3:} How to evaluate the quality of an abstract solver when solving real-life problems?

\subsection{Research Objective}
\paragraph*{Primary Objective: } To develop a framework which can allow researchers to solve a problem and evaluate their solvers as close to real-life enviornment as possible.
\paragraph*{RO1:} To extend the classical MAPF problem definititon to capture more real life nuances.
\paragraph*{RO2:} To develop an evaluation system wich involves using real robots, and simulated robots.
\section{Contribution to the Knowledge}
\section{Structure of the Report}
